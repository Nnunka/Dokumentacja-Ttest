	\newpage
\section{Testowanie}	%5
%Opisujemy testy, sprawdzamy czy nie generuje błędów.

\subsection{Testowanie latarki}

\begin{tabela}
	{Testowanie latarki}	%opis w spisie tabel
	{Testowanie latarki}	%opis przy tabeli
	{
		\begin{tabular}{|c|c|c|c|c|} \hline
			\textbf{lp} & \textbf{Zadania do przetestowania} & \textbf{Tak} & \textbf{Nie} \\ \hline
			1 & Latarka po naciśnięciu na guzik włączyła się & X & ~ \\ \hline
			2 & Po włączeniu latarki, guzik zmienia kolor na zielony & X & ~ \\ \hline
			3 & Wyświetlenie komunikatu o włączeniu latarki & X & ~ \\ \hline
			4 & Latarka po naciśnięciu na guzik wyłączyła się & X & ~ \\ \hline
			5 & Po wyłączeniu latarki, guzik zmienia kolor na czerwony & X & ~ \\ \hline
			6 & Wyświetlenie komunikatu o wyłączeniu latarki & X & ~ \\ \hline
		\end{tabular}	}
	\label{tab:tablica_latarka}
\end{tabela}

Obrazek \ref{rys:latarka} przedstawia zrzuty ekranu potwierdzające pomyślny przebieg testu.

\begin{figure}[!hbt]
	\begin{center}
		\includegraphics[angle=360, width=0.70\textwidth]{rys/punkt5/latarka.jpg}
		\caption{Przebieg testowania latarki}
		\label{rys:latarka}
	\end{center}
\end{figure}   

\newpage


\subsection{Testowanie trybu ciemnego}

\begin{tabela}
	{Testowanie trybu ciemnego}	%opis w spisie tabel
	{Testowanie trybu ciemnego}	%opis przy tabeli
	{
		\begin{tabular}{|c|c|c|c|c|} \hline
			\textbf{lp} & \textbf{Zadania do przetestowania} & \textbf{Tak} & \textbf{Nie} \\ \hline
			1 & Tryb ciemny po naciśnięciu na guzik włączył się & X & ~ \\ \hline
			2 & Po włączeniu trybu ciemnego, guzik zmienia kolor na zielony & X & ~ \\ \hline
			3 & Wyświetlenie komunikatu o włączeniu trybu ciemnego & X & ~ \\ \hline
			4 & Tryb ciemny po naciśnięciu na guzik wyłączył się & X & ~ \\ \hline
			5 & Po wyłączeniu trybu ciemnego, guzik zmienia kolor na czerwony & X & ~ \\ \hline
			6 & Wyświetlenie komunikatu o wyłączeniu trybu ciemnego & X & ~ \\ \hline
	\end{tabular}	}
	\label{tab:tablica_ciemny}
\end{tabela}

Obrazek \ref{rys:ciemny} przedstawia zrzuty ekranu potwierdzające pomyślny przebieg testu.

\begin{figure}[!hbt]
	\begin{center}
		\includegraphics[angle=360, width=0.70\textwidth]{rys/punkt5/ciemny.jpg}
		\caption{Przebieg testowania trybu ciemnego}
		\label{rys:ciemny}
	\end{center}
\end{figure}   

\newpage


\subsection{Testowanie czujnika zbliżeniowego}

\begin{tabela}
	{Testowanie czujnika zbliżeniowego}	%opis w spisie tabel
	{Testowanie czujnika zbliżeniowego}	%opis przy tabeli
	{
		\begin{tabular}{|c|c|c|c|c|} \hline
			\textbf{lp} & \textbf{Zadania do przetestowania} & \textbf{Tak} & \textbf{Nie} \\ \hline
			1 & Po zbiżeniu obiektu do czujnik zmienia się tekst informacyjny  & X & ~ \\ \hline
			2 & Po oddaleniu obiektu od czujnika zmienia się tekst informacyjny & X & ~ \\ \hline
	\end{tabular}	}
	\label{tab:tablica_zblizeniowy}
\end{tabela}

Obrazek \ref{rys:zblizeniowy} przedstawia zrzuty ekranu potwierdzające pomyślny przebieg testu.

\begin{figure}[!hbt]
	\begin{center}
		\includegraphics[angle=360, width=0.70\textwidth]{rys/punkt5/zblizeniowy.jpg}
		\caption{Przebieg testowania czujnika zbliżeniowego}
		\label{rys:zblizeniowy}
	\end{center}
\end{figure}   

\newpage


\subsection{Testowanie lokalizacji}

\begin{tabela}
	{Testowanie lokalizacji}	%opis w spisie tabel
	{Testowanie lokalizacji}	%opis przy tabeli
	{
		\begin{tabular}{|c|c|c|c|c|} \hline
			\textbf{lp} & \textbf{Zadania do przetestowania} & \textbf{Tak} & \textbf{Nie} \\ \hline
			1 & Po naciśnieciu na guzik wyświatla się adres obecnej lokalizacji  & X & ~ \\ \hline
			2 & Po naciśnieciu na guzik wyświatla się komunikat z współrzędnymi & X & ~ \\ \hline
	\end{tabular}	}
	\label{tab:tablica_gps}
\end{tabela}

Obrazek \ref{rys:gps} przedstawia zrzuty ekranu potwierdzające pomyślny przebieg testu.

\begin{figure}[!hbt]
	\begin{center}
		\includegraphics[angle=360, width=0.70\textwidth]{rys/punkt5/gps.jpg}
		\caption{Przebieg testowania lokalizacji}
		\label{rys:gps}
	\end{center}
\end{figure} 

\newpage


\subsection{Testowanie wifi}  

\newpage


\subsection{Testowanie dźwięku}

\begin{tabela}
	{Testowanie dźwięku}	%opis w spisie tabel
	{Testowanie dźwięku}	%opis przy tabeli
	{
		\begin{tabular}{|c|c|c|c|c|} \hline
			\textbf{lp} & \textbf{Zadania do przetestowania} & \textbf{Tak} & \textbf{Nie} \\ \hline
			1 & Po naciśnięciu na guzik z głośników słychać śpiew ptaków & X & ~ \\ \hline
	\end{tabular}	}
	\label{tab:tablica_dzwiek}
\end{tabela}

Obrazek \ref{rys:dzwiek} przedstawia zrzuty ekranu potwierdzające pomyślny przebieg testu.

\begin{figure}[!hbt]
	\begin{center}
		\includegraphics[angle=360, width=0.35\textwidth]{rys/punkt5/dzwiek.jpg}
		\caption{Przebieg testowania dźwięku}
		\label{rys:dzwiek}
	\end{center}
\end{figure} 

\newpage


\subsection{Testowanie mikrofonu}

\begin{tabela}
	{Testowanie mikrofonu}	%opis w spisie tabel
	{Testowanie mikrofonu}	%opis przy tabeli
	{
		\begin{tabular}{|c|c|c|c|c|} \hline
			\textbf{lp} & \textbf{Zadania do przetestowania} & \textbf{Tak} & \textbf{Nie} \\ \hline
			1 & Naciśnięcie na guzik "START" zaczyna nagrywanie & X & ~ \\ \hline
			2 & Po naciśnięciu na guzik "START" wyświetla się komunikat & X & ~ \\ \hline
			3 & Naciśnięcie na guzik "STOP" kończy nagrywanie & X & ~ \\ \hline
			4 & Po naciśnięciu na guzik "STOP" wyświetla się komunikat & X & ~ \\ \hline
			5 & Naciśnięcie na guzik "PLAY" oddtwarza nagranie & X & ~ \\ \hline
			6 &  Po naciśnięciu na guzik "PLAY" wyświetla się komunikat & X & ~ \\ \hline
	\end{tabular}	}
	\label{tab:tablica_mikrofon}
\end{tabela}

Obrazek \ref{rys:mikrofon} przedstawia zrzuty ekranu potwierdzające pomyślny przebieg testu.

\begin{figure}[!hbt]
	\begin{center}
		\includegraphics[angle=360, width=0.90\textwidth]{rys/punkt5/mikrofon.jpg}
		\caption{Przebieg testowania mikrofonu}
		\label{rys:mikrofon}
	\end{center}
\end{figure} 

\newpage


\subsection{Testowanie aparatu}

\begin{tabela}
	{Testowanie aparatu}	%opis w spisie tabel
	{Testowanie aparatu}	%opis przy tabeli
	{
		\begin{tabular}{|c|c|c|c|c|} \hline
			\textbf{lp} & \textbf{Zadania do przetestowania} & \textbf{Tak} & \textbf{Nie} \\ \hline
			1 & Po naciśnięciu na guzik "zrób zdjęcie" włącza się aplikacja aparatu & X & ~ \\ \hline
	\end{tabular}	}
	\label{tab:tablica_aparat}
\end{tabela}

Obrazek \ref{rys:aparat} przedstawia zrzuty ekranu potwierdzające pomyślny przebieg testu.

\begin{figure}[!hbt]
	\begin{center}
		\includegraphics[angle=360, width=0.90\textwidth]{rys/punkt5/aparat.jpg}
		\caption{Przebieg testowania aparatu}
		\label{rys:aparat}
	\end{center}
\end{figure} 

\newpage

