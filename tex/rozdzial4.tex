	\newpage
\section{Implementacja}		%4


Listing \ref{lst:listing-java} (s. \pageref{lst:listing-java}) przedstawia implementacje jednego z przycisków którego zadaniem jest otworzenie nowej strony oznaczonej w 5 linijce kodu po kliknięciu. 
\begin{lstlisting}[caption=Menu - Działanie Przycisków, label={lst:listing-java}, language=Java]
	button_1 = (Button) findViewById(R.id.button_1);
	button_1.setOnClickListener(new View.OnClickListener() {
		@Override
		public void onClick(View view) {
			Intent intent = new Intent(MainActivity.this, latarka.class);
			startActivity(intent);
		}
	});
\end{lstlisting}


Listing \ref{lst:listing-java2} (s. \pageref{lst:listing-java2}) przedstawia włączenie latarki. Jeśli telefon ma latarkę i mamy do niej dostęp, latarka zostanie uruchomiona i wyświetli się nam komunikat o włączeniu. Jeżeli nie, to dostaniemy odpowiednie powiadomienie o niepowodzeniu operacji. \\
W podobny sposób wygląda metoda z wyłączeniem latarki. W 6 linijce trzeba zamienić wartość na false oraz w 7 wierszu nadpisać wyświetlany komunikat na odpowiedni.
\begin{lstlisting}[caption=Latarka - Włączenie/wyłączenie latarki, label={lst:listing-java2}, language=Java]
	private void flashLightOn(){
		CameraManager cameraManager = (CameraManager) getSystemService(Context.CAMERA_SERVICE);
		try{
			assert cameraManager != null;
			String cameraId = cameraManager.getCameraIdList()[0];
			cameraManager.setTorchMode(cameraId, true);
			Toast.makeText(latarka.this, "Latarka wlaczona", Toast.LENGTH_SHORT).show();
		}
		catch(CameraAccessException e){
			Log.e("Camera Problem", "Nie mozna uruchomic latarki");
		}
	}
\end{lstlisting}

\newpage


Aby zaimplementować tryb ciemny potrzebujemy dostępu do modyfikowania interfejsu. Listing \ref{lst:listing-java3} (s. \pageref{lst:listing-java3}) pokazuje jak go uzyskać. 
\begin{lstlisting}[caption=Tryb Ciemny - Modyfikowanie interfejsu, label={lst:listing-java3}, language=Java]
	SharedPreferences sharedPreferences = getSharedPreferences("sharedPrefs", MODE_PRIVATE);
	final SharedPreferences.Editor editor = sharedPreferences.edit();
	final boolean isDarkModeOn = sharedPreferences.getBoolean("isDarkModeOn", false);
\end{lstlisting}


 Listing \ref{lst:listing-java4} (s. \pageref{lst:listing-java4}) przedstawia metodę odpowiedzialną za działanie trybu ciemnego. Domyślnie aplikacja korzysta z trybu jansego, co pokazuje kod od 5 do 10 linijki. Jednak po naciśnięciu na przycisku, zmieniamy kolorystyke aplikacji na ciemną (13-15 linijak), dodatkowo od 16 linijki zmieniamy przycisk na czerwony, aktualne logo na ciemne oraz wyświetlamy stosowny komunikat.
\begin{lstlisting}[caption=Tryb Ciemny - Włączenie/wyłączenie trybu ciemnego, label={lst:listing-java4}, language=Java]
	toggle_tryb_ciemny.setOnClickListener(new View.OnClickListener() {
		@Override
		public void onClick(View view) {
			if(isDarkModeOn){
				AppCompatDelegate.setDefaultNightMode(AppCompatDelegate.MODE_NIGHT_NO);
				editor.putBoolean("isDarkModeOn", false);
				editor.apply();
				toggle_tryb_ciemny.setImageResource(R.drawable.tryb_ciemny_off);
				logo.setImageResource(R.drawable.logo_white);
				Toast.makeText(tryb_ciemny.this, "Tryb ciemny wylaczony", Toast.LENGTH_SHORT).show();
			}
			else {
				AppCompatDelegate.setDefaultNightMode(AppCompatDelegate.MODE_NIGHT_YES);
				editor.putBoolean("isDarkModeOn", true);
				editor.apply();
				toggle_tryb_ciemny.setImageResource(R.drawable.tryb_ciemny_on);
				logo.setImageResource(R.drawable.logo_black);
				Toast.makeText(tryb_ciemny.this, "Tryb ciemny wlaczony", Toast.LENGTH_SHORT).show();
			}
		}
	});
\end{lstlisting}

\newpage


Listing \ref{lst:listing-java5} (s. \pageref{lst:listing-java5}) prezentuje uzyskanie dostępu do sensora, za pomocą którego sprawdzimy działanie czujnika zbliżeniowego.
\begin{lstlisting}[caption=Czujnik Zbliżeniowy - Dostęp do czujnika, label={lst:listing-java5}, language=Java]
	SensorManager sensorManager;
	sensorManager = (SensorManager)getSystemService(SENSOR_SERVICE);
	if(sensorManager!=null) {
		Sensor proximitySensor = sensorManager.getDefaultSensor(Sensor.TYPE_PROXIMITY);
		if(proximitySensor!=null) {
			sensorManager.registerListener(this, proximitySensor, sensorManager.SENSOR_DELAY_NORMAL);
		}
	}
\end{lstlisting}


Metoda onSensorChanged ukazana na listingu \ref{lst:listing-java6} (s. \pageref{lst:listing-java6}) sprawdza czy przy czujniku jest obiekt, zależnie od wyniku zostanie wyświetlony odpowiedni komunikat.
\begin{lstlisting}[caption=Czujnik Zbliżeniowy - Działanie, label={lst:listing-java6}, language=Java]
	@Override
	public void onSensorChanged(SensorEvent sensorEvent) {
		if(sensorEvent.sensor.getType()==Sensor.TYPE_PROXIMITY) {
			if(sensorEvent.values[0]==0) {
				((TextView)findViewById(R.id.sensor)).setText("Przy czujniku jest obiekt");
			} else {
				((TextView)findViewById(R.id.sensor)).setText("Przyloz obiekt do czujnika");
			}
		}
	}
\end{lstlisting}


Aby wdrożyć test aplikacji trzeba sprawdzić pozwolenie, w tym wypadku interesuje nas uprawnienie dostępu do lokalizacji, co obrazuje listing  \ref{lst:listing-java7} (s. \pageref{lst:listing-java7}). 
\begin{lstlisting}[caption=GPS - Dostęp do lokazlizacji, label={lst:listing-java7}, language=Java]
	if(ContextCompat.checkSelfPermission(gps.this, Manifest.permission.ACCESS_FINE_LOCATION) != PackageManager.PERMISSION_GRANTED) {
		ActivityCompat.requestPermissions(gps.this, new String[]{
			Manifest.permission.ACCESS_FINE_LOCATION
		}, 1000);
	}
\end{lstlisting}

\newpage


Metoda onLocationChanged przedstawiona jako listing \ref{lst:listing-java8} (s. \pageref{lst:listing-java8}) odpowiada za wyświetlenie powiadomienia push-up wyświetlającego długość i szerokość geograficzną obecnej lokalizacji oraz pobiera adres tej lokalizacji na podstawie długości i szerokości geograficznej, który wyświetla jako tekst.
\begin{lstlisting}[caption=GPS - Wyświetlanie lokalizacji, label={lst:listing-java8}, language=Java]
	@Override
	public void onLocationChanged(Location location) {
		Toast.makeText(this, ""+location.getLatitude()+", "+location.getLongitude(), Toast.LENGTH_SHORT).show();
		try {
			Geocoder geocoder = new Geocoder(gps.this, Locale.getDefault());
			List<Address> addresses = geocoder.getFromLocation(location.getLatitude(), location.getLongitude(), 1);
			String address = addresses.get(0).getAddressLine(0);
			text_location.setText(address);
		} catch (Exception e) {
			e.printStackTrace();
		}
	}
\end{lstlisting}


Listing \ref{lst:listing-java9} (s. \pageref{lst:listing-java9}) przedstawia implementacje MediaPlayer (linijka 1) do którego podłączamy dźwięk. Teraz w prosty sposób możemy wywołać dźwięk poprzez medote OnClick.
\begin{lstlisting}[caption= Dźwięk - Działanie z wykorzystaniem MediaPlayer, label={lst:listing-java9}, language=Java]
	final MediaPlayer sound = MediaPlayer.create(this, R.raw.dzwiek_audio);
	Button btn_dzwiek = findViewById(R.id.btn_dzwiek);
	btn_dzwiek.setOnClickListener(new View.OnClickListener() {
		@Override
		public void onClick(View view) {
			sound.start();
			Toast.makeText(dzwiek.this, "Odtwarzanie nagrania", Toast.LENGTH_SHORT).show();
		}
	});
\end{lstlisting}

\newpage


Aby test mikrofonu spełniał swoje zadanie trzeba sprawdzić autoryzację aplikacji. W tym wypadku uzyskujemy pozwolenie na nagrywanie audio, co przedstawia listing \ref{lst:listing-java10} (s. \pageref{lst:listing-java10}). 
\begin{lstlisting}[caption=Mikrofon - Dostęp do nagrywania audio, label={lst:listing-java10}, language=Java]
	private void getMicrophonePermission() {
		if(ContextCompat.checkSelfPermission(this, Manifest.permission.RECORD_AUDIO) == PackageManager.PERMISSION_DENIED) {
			ActivityCompat.requestPermissions(this, new String[] {Manifest.permission.RECORD_AUDIO}, MICROPHONE_PERMISSION_CODE);
		}
	}
\end{lstlisting}


Listing \ref{lst:listing-java11} (s. \pageref{lst:listing-java11}) przedstawia metodę z wykorzystaniem MediaRecorder który służy do nagrywania dźwięku lub obrazu. W tym wypadku MediaRecorder wykorzystany zostanie do nagrania dźwięku przez mikrofon w telefonie. Od 4 do 7 linijki określamy źródło dźwięku, format wyjściowy, zaznaczenie deskryptora pliku, definiowanie kodowania dźwięku.
\begin{lstlisting}[caption=Mikrofon - Włączenie nagrywania, label={lst:listing-java11}, language=Java]
	public void btnRecordPressed(View v) {
		try {
			mediaRecorder = new MediaRecorder();
			mediaRecorder.setAudioSource(MediaRecorder.AudioSource.MIC);
			mediaRecorder.setOutputFormat(MediaRecorder.OutputFormat.THREE_GPP);
			mediaRecorder.setOutputFile(getRecordingFilePath());
			mediaRecorder.setAudioEncoder(MediaRecorder.AudioEncoder.AMR_NB);
			mediaRecorder.prepare();
			mediaRecorder.start();
			
			Toast.makeText(this, "Nagrywanie rozpoczete", Toast.LENGTH_LONG).show();
		}
		catch (Exception e) {
			e.printStackTrace();
	}
\end{lstlisting}
	
\newpage
	
	
Po naciśnięciu przycisku "Stop" nagranie zostaje zakończone, co obrazuje listing \ref{lst:listing-java12} (s. \pageref{lst:listing-java12}). MediaRecorder kończy nagrywanie, a opcja realse użyta w 3 linijce nie pozwoli na ponowne uruchomienie pliku dźwiękowego w celu dogrania do niego dźwięku.
\begin{lstlisting}[caption=Mikrofon - Przerwanie nagrywania, label={lst:listing-java12}, language=Java]
	public void btnStopPressed(View v) {
		mediaRecorder.stop();
		mediaRecorder.release();
		mediaRecorder = null;
			
		Toast.makeText(this, "Nagrywanie zakonczone", Toast.LENGTH_LONG).show();
	}
\end{lstlisting}
	
Obiekt MediaPlayer wykorzystany w listingu \ref{lst:listing-java13} (s. \pageref{lst:listing-java13}) służy do obsługi odtwarzania plików audio i wideo. W 4 linijce kodu określamy skąd ma zostać pobrany plik, a następnie plik z dźwiękiem zostanie włączony.
\begin{lstlisting}[caption=Mikrofon - Odtworzenie nagrania, label={lst:listing-java13}, language=Java]
	public void btnPlayPressed(View v) {
		try {
			mediaPlayer = new MediaPlayer();
			mediaPlayer.setDataSource(getRecordingFilePath());
			mediaPlayer.prepare();
			mediaPlayer.start();
				
			Toast.makeText(this, "Odtwarzanie nagrania", Toast.LENGTH_LONG).show();
		}
		catch(Exception e){
			e.printStackTrace();
		}
	}
\end{lstlisting}

\newpage

Metoda przedstawiona na listingu \ref{lst:listing-java14} (s. \pageref{lst:listing-java14}), odpowiada za uruchomienie aparatu w telefonie po naciśnięciu przycisku o id btnCam. Od 6 do 8 linijki znajduję się kod opowiedzialny za uruchomienie aplikacji aparatu.
\begin{lstlisting}[caption=Aparat - Włączenie aparatu, label={lst:listing-java14}, language=Java]
	btnCam = findViewById(R.id.btnCam);
	btnCam.setOnClickListener(new View.OnClickListener() {
		@Override
		public void onClick(View view) {
			try {
				Intent intent = new Intent();
				intent.setAction(MediaStore.ACTION_IMAGE_CAPTURE);
				startActivity(intent);
			} catch (Exception e) {
				e.printStackTrace();
			}
		}
	});
\end{lstlisting}

Pętla switch (rozpoczynająca się w linijce 6) przedstawiona w listingu \ref{lst:listing-java15} (s. \pageref{lst:listing-java15}) odpowiada za informowanie użytkowanika o aktualnym statusie wifi. W linijce 7 sprawdzamy czy Wifi jest włączone, jeśli jest włączone - wyświetlamy odpowiedni tekst, w 10 linijce sprawdzamy czy wifi jest wyłączone na danym urządzeniu, jeśli tak zostaje wyświetlony adekwatny tekst.
\begin{lstlisting}[caption=Wifi - Sprawdzenie statusu Wifi, label={lst:listing-java15}, language=Java]
	private BroadcastReceiver wifiStateReceiver = new BroadcastReceiver() {
		@Override
		public void onReceive(Context context, Intent intent) {
			int wifiStateExtra = intent.getIntExtra(WifiManager.EXTRA_WIFI_STATE,
			WifiManager.WIFI_STATE_UNKNOWN);
			switch (wifiStateExtra) {
				case WifiManager.WIFI_STATE_ENABLED:
				wifiSwitch.setText("WiFi jest wlaczone, mozna pobrac informacje");
				break;
				case WifiManager.WIFI_STATE_DISABLED:
				wifiSwitch.setText("WiFi jest wylaczone, wlacz aby pobrac informacje");
				break;
			}
		}
	};
\end{lstlisting}

\newpage

Metoda przedstawiona na listingu \ref{lst:listing-java16} (s. \pageref{lst:listing-java16}) odpowiada za pobranie informacji o sieci wifi do której urządzenie jest obecnie podłączone oraz wyświetlenia tych informacji. Od 5 do 9 linijki mamy kolejno pobranie informacji o adresie IP telefonu, sformatowanie pobranego IP ma tekst, pobranie adresu MAC routera, SSID oraz wskaźnika mocy naszego połączenia. Pobrane informacje implementujmey jako string (od 11 do 15 linijki) i wyświetlamy jako tekst (linijka 17).
\begin{lstlisting}[caption=Wifi - Pobieranie informacji o Wifi, label={lst:listing-java16}, language=Java]
    public void getWifiInformation(View view) {
	WifiManager wifiManager = (WifiManager) getApplicationContext().getSystemService(WIFI_SERVICE);
	WifiInfo wifiInfo = wifiManager.getConnectionInfo();
	
	int ip = wifiInfo.getIpAddress();
	String ipAddress = Formatter.formatIpAddress(ip);
	String bssid = wifiInfo.getBSSID();
	String ssid = wifiInfo.getSSID();
	int rssi = wifiInfo.getRssi();
		
	String info =
	"\n Adres IP: " + ipAddress +
	"\n Adres MAC Routera: " + bssid +
	"\n SSID: " + ssid +
	"\n Wskaznik mocy: " + rssi;
	txtWifiInfo.setText(info);	
}
\end{lstlisting}


Kod ukazany w listingu \ref{lst:listing-java17} (s. \pageref{lst:listing-java17}) jest odpowiedzialny za pobranie marki i modeu urządzenia z którego obecnie korzystamy i wyświetlenie tej informacji jako tekst.
\begin{lstlisting}[caption=Wyniki - Pobranie marki i modelu urządzenia, label={lst:listing-java17}, language=Java]
	model = (TextView) findViewById(R.id.model);
	String stringBuildModel = "Marka i model: " + Build.MANUFACTURER + " " + Build.MODEL;
	model.setText(stringBuildModel);
\end{lstlisting}


